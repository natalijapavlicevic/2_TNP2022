\documentclass[a4paper]{article}

\usepackage{color}
\usepackage{url}
\usepackage[T2A]{fontenc} % enable Cyrillic fonts
\usepackage[utf8]{inputenc} % make weird characters work
\usepackage{graphicx}

\usepackage[english,serbian]{babel}
%\usepackage[english,serbianc]{babel} %ukljuciti babel sa ovim opcijama, umesto gornjim, ukoliko se koristi cirilica

\usepackage[unicode]{hyperref}
\hypersetup{colorlinks,citecolor=green,filecolor=green,linkcolor=blue,urlcolor=blue}

%\newtheorem{primer}{Пример}[section] %ćirilični primer
\newtheorem{primer}{Primer}[section]

	\title{Reklame na društvenim mrežama\\ \small{Seminarski rad u okviru kursa\\Tehničko i naučno pisanje\\ Matematički fakultet}}
	
	\author{Natalija Pavlićević (pavlicevicnatalija@gmail.com)\\ Teodora Maksimović (teamax18@gmail.com)\\ Branka Marković (bakymarkovic@gmail.com)\\ Mihailo Damnjanović (mihailo.damnjanović@gmail.com)}
	\date{15.~novembar 2022.}
	\maketitle
	
	\abstract{Ovo je apstrakt za ovaj seminarski rad.
	}

\tableofcontents

\newpage

\begin{document}
	
	\section{Prikupljanje podataka korisnika}
	\label{sec:podaci}
	Oglašivači sve više koriste personalizovano reklamiranje koje je prilagođeno potrošačima na osnovu podataka koji se tiču njihovih preferencija i ponašanja, a koje dobijaju prikupljanjem njihovih ličnih podataka. Ciljano reklamiranje je sve veća oblast interesovanja za i poslovne i istraživačke zajednice. Reklame u mobilnim uslugama i oglasi u aplikaciji predstavljaju oblast rasta u nastajanju, u kojoj ciljano reklamiranje postaje sve važniji izvor prihoda i za oglašivače i za reklamne kompanije. Ciljano reklamiranje se zasniva na analitici velikih podataka, gde lični podaci korisnika se prikupljaju i obrađuju za svrhe profilisanja i ciljanja.
	

	\section{Mišljenje korisnika o reklamama na internetu}
	\label{sec:misljenje}
	Kroz istraživanje o mišljenju korisnika za reklame na internetu, došli smo do zaključka da se mišljenja razlikuju i da ima korisnika kojima smetaju reklame, a ima i onih koji vole da kliknu i vide šta im se ponudi u reklami.
	Same reklame korisnicima mogu da smetaju prilikom korišćenja nekih aplikacija ili gledanja filmova,kao i na youtube-u dok slušaju muziku jer reklame samo iskaču.To korisnicima kvari događaj i oni samo preskaču te reklame.
	Međutim, naravno da ima korisnika i koji vole kad im iskoči neka reklama, naročito ako oni u tom trenutku tragaju za tim proizvodima koji se reklamiraju, oni klikću na reklame.
	\subsection{Da li korisnici veruju reklamama na internetu?}
	\label{subsec:veovanje_reklamama}
	Dosta korisnika veruje reklamama na internetu ali samo onim koji su provereni ili se pronalaze u njima.Naravno ima određeni broj korisnika koji ne veruje reklamama i oni misle da su te reklame samo prevare koje privlače korisnike da kupe proizvode koji nisu tako dobri kao na reklama.U nastavku možete videti sliku \ref{fig:verovanje_reklamama} koliko procenata veruje,a koliko ne.
	\begin{verbatim}
		\usepackage{graphicx}
	\end{verbatim}
	
	\begin{figure}[h!]
		\begin{center}
			\includegraphics[scale=0.75]{verovanje_reklamama.jpg}
		\end{center}
		\caption{verovanje_reklamama}
		\label{fig:verovanje_reklamama}
	\end{figure}
	Ovo istraživanje izvršeno je od strane Social Serbia 2020.
	\subsection{Koliko cesto korisnici klikcu na reklame?}
	\label{subsec:klik_na_reklamu}
	Zanimljiva činjenica kod ovoga jeste to da je veći procenat muškaraca koji često klikcu na reklame nego žena. U nastavku možete videti sliku \ref{fig:klik_na_reklamu} koliko često korisnici klikću na reklame.
	\begin{verbatim}
	 \usepackage{graphicx}
	\end{verbatim}
	
	\begin{figure}[h!]
		\begin{center}
			\includegraphics[scale=0.55]{klik_na_reklamu.jpg}
		\end{center}
		\caption{klik_na_reklamu}
		\label{fig:klik_na_reklamu}
	\end{figure}
	Takođe i ovo istraživanje izvršeno je od strane Social Serbia 2022.
	\section{Doprinos od reklama}
	\label{sec:doprinos}
	Mnogi proizvođači se pitaju da li reklame doprinose prodaji ili ne.Doprinos je uglavnom veliki ukoliko se reklamirate na pravi način.Reklame na internetu u današnje vreme privuku najviše kupaca.
	\subsection{Da li se treba reklamirati?}
	\label{subsec:potrebna_reklama}
		Šta više uopšte se ne treba postavljati ovo pitanje.Prema istaraživanjima veliki procenat ljudi reklamira proizvode preko interneta, zato što su ljudi postali lenji i sve rade preko interneta.U nastavku možete videti listu o procentima doprinosa reklamiranja na internetu:
		1.84\% marketing stručnjaka se reklamira na nekoj društvenoj mreži
		2.83\% njih insistira da su društvene mreže bitne za njihov biznis
		3.Broj preduzetnika koji tvrde da je Facebook neophodan za njihov biznis je skocio na 75\%
		4.Društvene mreže su dovele do značajnog porasta stopi koverzije u poređenju sa tradicionalnim marketingom 
		Ovo istraživanje izvršeno je od stane hubspot marketing statistics.
\end{document}